\documentclass[a4paper, 12pt]{article}
\usepackage[top=1cm, bottom = 2cm, left = 2cm, right = 2cm]{geometry}
\usepackage[utf8]{inputenc}
\usepackage[brazil]{babel}
\usepackage{listings}
\usepackage[framed, numbered]{matlab-prettifier}
\usepackage[T1]{fontenc}
\usepackage{indentfirst}
\usepackage{graphicx}
\usepackage{epstopdf}
\usepackage{float}
\usepackage{amsmath}
\usepackage{amssymb}
\usepackage{systeme}

\title{Exercício 1 - Aula 4 \\ EET-01}

\author{
  Igor Caldeira Magalhães\\igorcmag@gmail.com
}
\date{08 de maio de 2020}

\begin{document}
\maketitle
\section{Enunciado}

\subsection{a)}
Derivar (demonstrar) a desigualdade de Cauchy-Schwarz.

\subsection{b)} 
Derivar (demonstrar) a relação de Euler.

%O exercício deve ser entregue: Documento simples deve ser apresentado, contendo o seguinte conteúdo:
%Nome da disciplina;
%Nome do aluno;
%Enunciado do exercício;
%Figuras obtidas - Com breve descrição: A Figura xx mostra ...
%Código .m e breve descrição do código matlab (octave);

\section{Solução}

\subsection{a)}

Para todos $a_i, b_i$ e $x$ reais vale

$$(a_1x+b_1)^2 + (a_2x+b_2)^2 + ... + (a_nx+b_n)^2 \geqslant 0. $$

Reagrupando, temos

$$(a_1^2+a_2^2+ ... +a_n^2)x^2 + 2(a_1b_1 + a_2b_2 + ... + a_nb_n)x + (b_1^2 + b_2^2 + ... + b_n^2) \geqslant 0$$

independentemente de $x$. Logo, o determinante da equação deve ser não positivo, ou seja, 

$$4(a_1b_1 + a_2b_2 + ... + a_nb_n)^2-4(a_1^2+a_2^2+ ... +a_n^2)(b_1^2 + b_2^2 + ... + b_n^2)\leqslant 0$$
$$\therefore (a_1^2+a_2^2+ ... +a_n^2)(b_1^2 + b_2^2 + ... + b_n^2) \geqslant (a_1b_1 + a_2b_2 + ... + a_nb_n).$$

Se $a = (a_1,a_2,...,a_n)$ e $b = (b_1,b_2,...,b_n)$, então a desigualdade pode ser escrita como
$$\mid a\mid\mid b\mid \geqslant \mid ab\mid$$

\subsection{b)}

A expansão de $e^x$ em série de Taylor em torno de a = 0 é dada por
$$e^x = \sum_{n=0}^{\infty}\frac{x^n}{n!}$$
com raio de de convergência infinito. A fórmula é válida para $x$ real, mas vamos verificar o que obteríamos para $ix$, com $x$ real.
$$e^{ix} = \sum_{n=0}^{\infty}\frac{(ix)^n}{n!}=\sum_{n=0}^{\infty}\frac{(-1)^n\cdot x^{2n}}{(2n)!}+i\sum_{n=1}^{\infty}\frac{(-1)^{n-1}\cdot x^{2n-1}}{(2n-1)!}$$
Observe que as parcelas do lado direito da equação são, respectivamente, as expansões de $cosx$ e $senx$. Motivados por isso, \textbf{definimos} a exponencial de um número complexo $z = x + y\cdot i$, com $x$ e $y$ reais, como 

$$e^z=e^{x + y\cdot i}=e^x(cosy + i\cdot seny)$$ 
\end{document}